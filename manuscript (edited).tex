\documentclass[12pt, letterpaper, titlepage]{article}

\usepackage{amsmath}
\usepackage{booktabs}
\usepackage{amsthm}
\usepackage{graphicx}
\usepackage[margin=1in]{geometry}
\usepackage{hyperref}
\hypersetup{colorlinks = true, linkcolor = blue, citecolor=blue, urlcolor = blue}
\usepackage{natbib}
\usepackage{enumitem}
\usepackage{setspace}
\usepackage{graphicx}

\usepackage[pagewise]{lineno}
%\linenumbers*[1]
% %% patches to make lineno work better with amsmath
\newcommand*\patchAmsMathEnvironmentForLineno[1]{%
 \expandafter\let\csname old#1\expandafter\endcsname\csname #1\endcsname
 \expandafter\let\csname oldend#1\expandafter\endcsname\csname end#1\endcsname
 \renewenvironment{#1}%
 {\linenomath\csname old#1\endcsname}%
 {\csname oldend#1\endcsname\endlinenomath}}%
\newcommand*\patchBothAmsMathEnvironmentsForLineno[1]{%
 \patchAmsMathEnvironmentForLineno{#1}%
 \patchAmsMathEnvironmentForLineno{#1*}}%

\AtBeginDocument{%
 \patchBothAmsMathEnvironmentsForLineno{equation}%
 \patchBothAmsMathEnvironmentsForLineno{align}%
 \patchBothAmsMathEnvironmentsForLineno{flalign}%
 \patchBothAmsMathEnvironmentsForLineno{alignat}%
 \patchBothAmsMathEnvironmentsForLineno{gather}%
 \patchBothAmsMathEnvironmentsForLineno{multline}%
}

% control floats
\renewcommand\floatpagefraction{.9}
\renewcommand\topfraction{.9}
\renewcommand\bottomfraction{.9}
\renewcommand\textfraction{.1}
\setcounter{totalnumber}{50}
\setcounter{topnumber}{50}
\setcounter{bottomnumber}{50}

\newcommand{\jy}[1]{\textcolor{blue}{JY: #1}}
\newcommand{\eds}[1]{\textcolor{red}{EDS: (#1)}}


\title{Thesis Topic}

\author{Owen Fiore\\
%   \href{mailto:owen.fiore@uconn.edu}
% {\nolinkurl{owen.fiore@uconn.edu}}\\
  Elizabeth Schifano\\
  Jun Yan\\[1ex]
  Department of Statistics, University of Connecticut\\
}
\date{}

\begin{document}
\maketitle

\doublespace

\begin{abstract}
Abstact/Introduction

\bigskip
\noindent\sc{Keywords}:
nonparametric bootstrap;
parametric bootstrap;
\end{abstract}

\section{Introduction}
\label{sec:intro}

Suppose that the readius of a cirlce is $r$. Then it's area is
\begin{equation}
 \label{eq:area}
 \pi r^2
\end{equation}

Equation~\eqred{eq:area} is something silly

Sometimes I don't want an equation to be numbered such as this one:

Here is a table:
Table~\ref{tab:rv} summarizes some concepts
\begin{table}[ht]
 \caption{This is my first table}
 \label{tab:rv}
 
Figure~ref{fig:cars} shows the distance against the speed
\begin{figure}
 \centering
 \includegraphics[width=\textwidth]{cars.pdf}
 
 

The Kolmogorov-Smirnov (KS) test is one of the most popular goodness-of-fit 
tests for comparing a sample with a hypothesized parametric distribution.
Let $X_1, ..., X_n$ be a random sample of size~$n$ from a continuous
distribution. The null hypothesis $H_0$ is that $X_i$'s follow distribution~$F$.
Let $F_n(t) = \sum_{i=1}^n I(X_i \le t) / n$ be the empirical cumulative
distribution function of the sample, where $I(\cdot)$ is the indicator
function. The KS test statistic is
\begin{equation}
  \label{eq:ks_standard}
  D_n = \sup_x | F_{n}(x) - F(x) |.
\end{equation}
The asymptotic distribution of $D_n$ under $H_0$ is independent of the
distribution $F$. As $n \to \infty$, $\sqrt{n} D_n$ converges in distribution to
the supremum of standard Brownian bridge \citep{kolmogorov1933sulla}. For large
samples, the tests can be performed with a table \citep{smirnov1948table}.
Critical values for small samples ($n \le 35$) has also been given
\citep{Massey}. The KS test is available in popular 
statistical software packages, such as function \texttt{ks.test} in R package
\textsf{stats} \citep{R, Marsaglia}.


The standard one-sample KS test applies to independent data with a continuous
hypothesized distribution that is completely specified. In practice, however, it
has often been applied without realizing that one or more of these assumptions
do not hold. For example, \citet{Noether} demonstrated the conservativeness of
the KS test when applied to discontinuous distributions. The null distribution
of the KS statistic is no longer distribution free. Computing the
exact and asymptotic distribution of $D_n$ is challenging. Fortunately, the null
distribution of the KS statistic when the underlying distribution is
discontinuous has been efficiently addressed by \citet{Dimitrova} with an
companion R package \textsf{KSgeneral}. Although a common misuse of KS test, the
issue with discontinuous or data is not our focus.


The standard KS test is not applicable when the hypothesized distribution 
contains fitted parameters. \citet{Steinskog} ``discovers'' the change in power 
when using estimated parameters and stresses caution in using the KS test in 
such ways. \citet{Lilliefors} shows for the normal distribution that using the 
standard table when values of the mean and standard deviation are estimated 
yields extremely conservative results. This is supported by \citet{Capasso}, 
who concludes that failing to re-estimate the parameters may lead to wrong, 
overly-conservative approximations to the distributions of goodness-of-fit test 
statistics based on the empirical distribution function. \citet{Capasso} also 
notes that the impact of this mistake may turn out to be dramatic and does not 
vanish as the sample size increases. Remedies are provided by \citet{Babu} and 
\citet{Genest} in the form of bootstrap. \citet{Babu} details the bootstrap
procedure for goodness-of-fit tests and notes that both parametric and 
non-parametric procedures lead to correct asymptotic levels, however there is a 
correction required for the non-parametric case. \citet{Genest} provides 
validity for using parametric bootstrap with various goodness-of-fit tests.

In the case of serially dependent data, \citet{Durilleul} demonstrates that the 
KS statistic is too liberal for medium-to-high positive autocorrelation values. 
\citet{Durilleul} also shows that for negative autocorrelation values, the 
behavior is asymmetrical with respect to positive values. For remedies, 
\citet{Weiss} provides a procedure that is applicable specifically for data 
modeled by the second-order auto-regressive (AR) process where the parameters 
are known. \citet{Lanzante} tests various strategies for dealing with temporal 
dependence and concludes that a test based on Monte-Carlo simulations performed 
the best. %We propose a parametric bootstrap procedure involving copulas to 
%account for dependence.

The contribution of this paper is a demonstration of misuses of the one-sample
KS test in three scenarios and their remedies in practice. The scenarios are 
where:
1) the hypothesized distribution has unspecified parameters;
2) the data are serially dependent; and
3) a combination of the first two scenarios. 
In each scenario, the misuse is performed and the impacts are shown. Then, a
remedy is detailed and performed alongside the misuse to show its positive 
effects. In order to set up the demonstrations, simulated data is used 
throughout. The remedies are also performed on various families of 
distributions.

The rest of the paper is organized as follows. Section~\ref{sec:fitted}
investigates the scenario where the hypothesized distribution has unspecified
parameters. Both parametric and nonparametric bootstrap are available to fix the
issue. Section~\ref{sec:dependence} investigates the scenario where the data of 
the empirical distribution is serially dependent. A bootstrap procedure 
employing copulas to account for dependence is proposed as a working solution. 
Section~\ref{sec:fittedwithdependence}
explores the case where a combination of the first two scenarios occurs. The 
copula procedure can be adjusted for the use of fitted parameters as a remedy. 
Section~\ref{sec:conclusion} concludes with a discussion.

\section{Unspecified Parameters}
\label{sec:fitted}

The null distribution of the KS statistic changes when the hypothesized
distribution contains fitted parameters. 
\eds{what is the null distribution when not using fitted parameters}.  
In this scenario, the null hypothesis
is $H_0$: the random sample $X_1, \ldots, X_n$ comes from a continuous
distribution $F_{\theta}$ with unspecified parameter vector $\theta$.
Let $\hat\theta_n$ be an estimator of $\theta$, which could be, for example, a
maximum likelihood estimator or moment estimator. The test statistic is
\begin{equation}
  \label{eq:ks_fitted}
  D = \sup_x | F_n(x) - F_{\hat\theta_n}(x) |.
\end{equation}
Since $F_{\hat\theta_n}$ is not the same as the true data generating
$F_\theta$, the null distribution of $D$ obtained in existing implementations in
software packages which assumes completely known $F_\theta$ no longer applies
\citep{Steinskog}.


To demonstrate the consequences of this problem, a simulation is performed. A 
random sample $X_1, \ldots, X_n$ is generated from the standard normal 
distribution with sample size $n = 100$. The p-values of $1000$ replicate tests
 of normality  
are displayed in the Naive plot of Figure~\ref{fig:hist_fitted}. 
\eds{how are these p-values calculated?}
Each KS test 
was performed using fitted parameters, i.e., the hypothesized distribution is 
$N(\bar X, s^2)$ where $\bar X$ is the sample mean and $s^2$ is the sample 
variance. Since the data is truly generated from a standard normal distribution,
 %with 
%seemingly all assumptions met, 
a uniform distribution of $U(0, 1)$ would naively be expected for the p-values.
However this should only be expected when the KS test assumptions hold, and they
 do not due to using fitted parameters. Therefore, 
there is notable deviation from the uniform distribution. 


To fix the problem, parametric bootstrap can be used to approximate the null
distribution of the testing statistic. 
\begin{enumerate}
  \item 
    Draw a random sample $X_1^*,...,X_n^*$ from the fitted distribution 
    $F_{\hat\theta_n}$
  \item 
    Fit $F_\theta$ to the sample and obtain estimated $\hat\theta_n^*$
  \item
    Obtain the empirical distribution function $F_n^*$ of $X_1^*, \ldots, 
    X_n^*$.
  \item 
    Calculate bootstrap KS statistic
    \[
      D^* = \sup_x | F_n^* (x)- F_{\hat\theta_n}^*(x) |.
    \]
  \item
    Repeat the previous steps a large number $B$ times and use the empirical 
    distribution of $D^*$ to approximate the null distribution of the observed 
    statistic.    
\end{enumerate}


Nonparametric bootstrap can also be used to approximate the null distribution 
of the testing statistic. The procedure is similar, however the resampling is 
performed with the empirical distribution instead of the fitted parametric
distribution and there is a correction for bias that is required
\citep{Babu}.
\begin{enumerate}
  \item 
    Draw a random sample $X_1^*,...,X_n^*$ from the empirical distribution $F_n$
    with replacement
  \item 
    Fit $F_\theta$ to the sample and obtain estimated $\hat\theta_n^*$
  \item
    Obtain the empirical distribution function of the random sample $F_n^*$
  \item 
    Calculate bootstrap KS statistic
    \[
      D^* = \sup_x | F_n^* (x)- F_{\hat\theta_n}^*(x) - B_n(x) |.
    \]
    where $B_{n}(x) = \sqrt{n}(F_{n}(x) - F_{\hat\theta_n}(x))$ is the known 
    bias term \citep{Babu}
  \item
    Repeat the previous steps a large number $B$ times and use the empirical 
    distribution of $D^*$ to approximate the null distribution of the observed 
    statistic.
\end{enumerate}
The p-value can be calculated by counting the number of bootstrap KS 
statistics greater than or equal to the observed KS statistic, and then dividing 
by the number of bootstrap samples. Figure~\ref{fig:hist_fitted} displays the 
results of from our simulations. We would expect the distribution of p-values 
to be uniform in the case where the KS test holds its size. It is clear from the 
figure that both parametric and nonparametric bootstrap processes correct for 
the problem of fitted parameters. The plots for the bootstrapped p-values appear
to be $U(0,1)$, unlike the naive p-values.

\begin{figure}[tbp]
  \centering
  \includegraphics[width=\textwidth]{hist_fitted}
  \caption{The Naive plot (left) is the histogram of p-values from the standard 
  KS test with fitted parameters. The Parametric plot (middle) is the histogram
  of p-values from implementing parametric bootstrap. The Babu plot (right) is 
  the histogram of p-values from implementing nonparametric bootstrap with a 
  correction for bias. In each case, $1000$ replicate tests were performed on 
  the standard normal distribution with sample size $n = 100$. $B = 1000$ 
  bootstrap samples are obtained for each test. \eds{perhaps use density hist vs
	freq. hist (here and elsewhere)?}}
  \label{fig:hist_fitted}
\end{figure}


\section{Serially Dependent Data}
\label{sec:dependence}

The KS test also displays issues in the case of dependent data. As mentioned, an 
assumption of the test is that the data is independent. Real data, however,
is often temporally or spatially dependent and the results of a goodness-of-fit 
test would be valuable. When the KS test is performed on dependent data it 
performs poorly. In this situation, the null hypothesis is $H_0$: the random 
sample $X_1, \ldots, X_n$ come from a continuous distribution $F$ where $F$ has 
completely specified parameters. The test statistic is the same as 
Equation~\eqref{eq:ks_standard}. However, since the data is dependent, the null 
hypothesis is wrong and must be corrected. 
\eds{null hypothesis or null distribution? or both?}
This is demonstrated with a 
simulation in Figure~\ref{fig:hist_correlation}. Data is generated from a 
first-order autoregressive model (AR(1)) with a standard normal distribution. 
P-values are gathered from $1000$ replicate tests for different levels of the 
AR coefficient $\psi$, varying from $(-3,3)$. 
\eds{how are these p-values calculated?}
In the presence of serially 
dependent data, the distribution of p-values no longer follows $U(0, 1)$ as 
would be expected of a valid test. The results echo those of \citet{Durilleul} 
that the KS statistic is too liberal in the presence of positive
autocorrelation, and that the behavior is asymmetrical for negative
 autocorrelation.

\begin{figure}[tbp]
  \centering
  \includegraphics[width=\textwidth]{hist_correlation}
  \caption{Each histogram is a plot of the p-values of standard KS tests
  performed on dependent data simulated from an AR(1) process of the standard 
  normal distribution. The titles represent the various levels of the AR 
  coefficient $\psi$ tested. The sample size is $n = 100$ and $1000$ replicate 
  tests were performed for each AR coefficient.}
  \label{fig:hist_correlation}
\end{figure}

In order to correct this, we can employ a parametric bootstrap procedure which
assumes a working serial dependence structure through copulas. A copula is a
multivariate distribution with standard uniform marginal distributions, which
completely characterizes the dependence structure of a multivariate
distribution \citep{Copula, Hofert}. For simplicity, we assume a normal copula 
with an AR(1) structure to characterize the serial dependence of the 
observations. The AR(1) parameter $r$ of the normal copula is set to match the 
sample serial Spearman's rho of the observed data. 
\eds{What is the difference between $\psi$ and $r$?}
The procedure is as 
follows.
  
\begin{enumerate}
\item
  Generate $Z_1, \ldots, Z_n$ from an AR(1) process with autocorrelation
  coefficient $r$ such that the $Z_i$'s are $N(0, 1)$ variables.
\item
  Form a bootstrap sample $X_i^* = F^{-1} [\Phi(Z_i)]$, where $\Phi$ is the
  distribution function of $N(0, 1)$, $i = 1, \ldots, n$, whose first-order 
  sample Spearman's rho matches that of the observed data.
\item
  Obtain the empirical distribution function $F_n^*$ of the bootstrap sample
  $X_1^*, \ldots, X_n^*$.
\item 
  Calculate bootstrap KS statistic
  \[
    D^* = \sup_x \lvert F_n^* (x)- F(x) \rvert.
  \]
\item
  Repeat the previous steps a large number $B$ times and use the empirical
    distribution of the $B$ test statistics to approximate
    the null distribution of the observed statistic.      
\end{enumerate}

Throughout the simulation we assume a working AR(1) dependence structure 
regardless of the true dependence. If the true dependence is indeed a normal 
copula with an AR(1) structure, this method is exact. When the true dependence 
is not AR(1), it may still give a reasonable approximation that can be useful 
for practical purposes. This is demonstrated with different dependence 
structures in Figure~\ref{fig:hist_ma1_arma_ar2_D}.

\begin{figure}[tbp]
  \centering
  \includegraphics[width=\textwidth]{hist_ar1_D}
  \caption{The Naive plot (left) is the histogram of p-values from the 
  standard KS test. The Copula plot (right) is the histogram of p-values from 
  implementing parametric bootstrap with copulas for dependence. In each case, 
  $1000$ replicate tests were performed with the data generated from an AR(1) 
  process of the standard normal distribution with $\psi = .5$. The sample size 
  is $n = 100$ and $B = 1000$ bootstrap samples are obtained for each test.}
  \label{fig:hist_ar1_D}
\end{figure}

Figure~\ref{fig:hist_ar1_D} displays the results of the copula remedy for 
dependent data. The data is generated from the standard normal distribution with
an AR(1) dependence structure where $\psi = .5$. The copula used is to model 
dependence is the normal copula. The distribution of p-values is again expected
to be $U(0, 1)$ for a valid test. The Naive plot is clearly not uniform and 
reinforces the results shown in Figure~\ref{fig:hist_correlation}. The Copula 
plot, which implements the aforementioned procedure to correct for dependence
in the data, appears to be uniform and restores the size of the KS test.


\begin{figure}[tbp]
  \centering
  \includegraphics[width=\textwidth]{hist_ma1_arma_ar2_D}
  \caption{The Naive plots are the histograms of p-values from the standard KS 
  test. The Copula plots are the histograms of p-values from parametric 
  bootstrap with copulas for dependence. The data is generated from an MA(1) 
  process (left) with $\theta = .5$, ARMA(1, 1) process (middle) with $\psi = 
  .5$ and $\theta = .3$, and AR(2) process (right) with $\psi = (.5, .3)$ of the 
  standard normal distribution. In each case, $1000$ replicate tests were 
  performed with sample size $n = 100$ and $B = 1000$ bootstrap samples.}
  \label{fig:hist_ma1_arma_ar2_D}
\end{figure}

The procedure detailed in this section can also provide a reasonable 
approximation in cases where the true dependence structure is not far from 
AR(1). Figure~\ref{fig:hist_ma1_arma_ar2_D} shows the distribution of p-values 
for dependence structures MA(1), ARMA(1, 1), and AR(2). As expected, 
naively performing the KS test without correcting for dependence provides plots 
of p-values that deviate from the uniform distribution. In the case of MA(1) 
and ARMA(1, 1), the true dependence structure is close enough to our 
assumption of AR(1) that the bootstrap procedure provides a reasonable 
approximation. However, the AR(2) copula plot shows the limitation of this 
technique as no AR(1) process can approximate an AR(2) process unless the
second-order coefficient is close to zero.

\begin{figure}[tbp]
  \centering
  \includegraphics[width=\textwidth]{hist_gamma_gev_D}
  \caption{The Naive plots are the histograms of p-values from the standard KS 
  test. The Copula plots are the histograms of p-values from parametric 
  bootstrap with copulas for dependence. The data is generated from $Gamma(3, 
  1)$ (left) and $GEV(0, .2, 1)$ (right) with a correlation coefficient of $r = 
  0.5$. In each case, $1000$ replicate tests were performed with sample size $n 
  = 100$ and $B = 1000$ bootstrap samples.}
  \label{fig:hist_gamma_gev_D}
\end{figure}

To further demonstrate the effectiveness of the procedure, we perform similar 
simulations on other families of distributions. 
Figure~\ref{fig:hist_gamma_gev_D} displays the results of tests using dependent 
data generated from the gamma and the generalized extreme value distributions.
The procedure corrects the distribution of p-values for both distributions and 
is shown to be applicable regardless of the family of distribution tested.


\section{Unspecified Parameters and Serially Dependent Data}
\label{sec:fittedwithdependence}

The procedure demonstrated in Section~\ref{sec:dependence} works when the data 
is dependent and the hypothesized distribution is completely specified. However,
this is not practical. In practice we do not know the parameters of the 
hypothesized distribution. Therefore, it is valuable to have a procedure that
corrects for both fitted parameters and serially dependent data. The remedy 
using copulas can be modified to be effective in the case where both assumptions
must be violated. The null hypothesis is equivalent to Section~\ref{sec:fitted} 
and the test statistic is equal to Equation~\eqref{eq:ks_fitted}. The bootstrap
procedure in Section~\ref{sec:fitted} is no longer valid because the serial
dependence is not accounted for. Let $r$ be the AR(1) coefficient of the normal 
copula that matches the sample first-order Spearman's rho of the observed data 
as obtained by the \texttt{iRho} function in the \textsf{copula} package 
\citep{Copula}. We propose the following bootstrap procedure to assess the 
significance of the observed KS statistic.

\begin{enumerate}
\item
  Generate $Z_1, \ldots, Z_n$ from an AR(1) process with autocorrelation
  coefficient $r$ such that the $Z_i$'s are $N(0, 1)$ variables.
\item
  Form a bootstrap sample $X_i^* = F^{-1}_{\hat\theta_n} [\Phi(Z_i)]$,
  $i = 1, \ldots, n$, whose first-order sample Spearman's rho matches that of
  the observed data.
\item
  Fit $F_\theta$ to the sample $X_1^*, \ldots, X_n^*$ and obtain estimator 
  $\hat\theta_n^*$
\item
  Obtain the empirical distribution function $F_n^*$ of the bootstrap sample
  $X_1^*, \ldots, X_n^*$.
\item 
  Calculate bootstrap KS statistic
  \[
    D^* = \sup_x \lvert F_n^* (x)- F_{\hat\theta_n^*}(x) \rvert.
  \]
\item
  Repeat the previous steps a large number $B$ times and use the empirical
    distribution of the $B$ test statistics to approximate
    the null distribution of the observed statistic.      
\end{enumerate}

\begin{figure}[tbp]
  \centering
  \includegraphics[width=\textwidth]{hist_ar1_FD}
  \caption{The Super Naive plot (left) is the histogram of p-values from the 
  standard KS test with fitted parameters. The Naive Parametric plot (middle) is
  the histogram of p-values from implementing parametric bootstrap. The Copula 
  plot is the histogram of p-values from implementing parametric bootstrap with 
  copulas for dependence and correcting for fitted parameters. In each case, 
  $1000$ replicate tests were performed with the data generated from an AR(1) 
  process of the standard normal distribution with $\psi = .5$. The sample size 
  is $n = 100$ and $B = 1000$ bootstrap samples are obtained for each test.}
  \label{fig:hist_ar1_FD}
\end{figure}

Figure~\ref{fig:hist_ar1_FD} shows the results of our simulation on data 
simulated from an AR(1) process. As should be expected, naively fitting 
parameters while providing no adjustment for dependent data invalidates the KS 
test. As well, only using the parametric bootstrap remedy for fitted parameters 
from Section~\ref{sec:fitted} seems to provide weaker results than the 
copula remedy presented above. The results show that copula remedy generates 
uniform p-values and restores the size of the KS test.

\begin{figure}[tbp]
  \centering
  \includegraphics[width=\textwidth]{hist_ma1_arma_ar2_FD}
  \caption{The Super Naive plots (left) are the histograms of p-values from the 
  standard KS test with fitted parameters. The Naive Parametric plots (middle) 
  are the histograms of p-values from implementing parametric bootstrap. The 
  Copula plots (bottom) are the histograms of p-values from implementing 
  parametric bootstrap with copulas for dependence and correcting for fitted 
  parameters. The data is generated from an MA(1) process (left) with 
  $\theta = .5$, ARMA(1, 1) process (middle) with $\psi = .5$ and $\theta = .3$,
  and AR(2) process (right) with $\psi = (.5, .3)$ of the standard normal 
  distribution. In each case, $1000$ replicate tests were performed with sample 
  size $n = 100$ and $B = 1000$ bootstrap samples.}
  \label{fig:hist_ma1_arma_ar2_FD}
\end{figure}

Similar to Section~\ref{sec:dependence}, the copula approach is not a complete 
solution. Regardless of the true dependence in the data we assume an AR(1) 
dependence structure by taking the lag-1 sample auto-spearman rho. However, we 
can show that as long as the AR(1) assumption is a close approximation, the 
correction still provides a reasonable approximation that can be useful for 
practical purposes. Figure~\ref{fig:hist_ma1_arma_ar2_FD} shows the results of the 
procedure performed on data generated with dependence structures of  MA(1), 
ARMA(1, 1), and AR(2). Naively fitting parameters and not adjusting for 
dependence clearly deviates from a uniform distribution of p-values. Parametric 
bootstrap provides some correction but does not account for dependence, so 
therefore the results of the copula remedy are more accurate and favorable.
In the case of MA(1) and ARMA(1, 1), the true dependence appears close enough 
to our assumption of AR(1) that the results are reasonable. AR(2) however
appears to be just far enough from our assumption, showing a limitation of the 
procedure.

\begin{figure}[tbp]
  \centering
  \includegraphics[width=\textwidth]{hist_gamma_gev_FD}
  \caption{The Super Naive plots (left) are the histograms of p-values from the 
  standard KS test with fitted parameters. The Naive Parametric plots (middle) 
  are the histograms of p-values from implementing parametric bootstrap. The 
  Copula plots are the histograms of p-values from implementing parametric 
  bootstrap with copulas for dependence and correcting for fitted parameters. 
  The data is generated from $Gamma(3, 1)$ (left) and $GEV(0, .2, 1)$ (right) 
  with a correlation coefficient of $r = 0.5$. In each case, $1000$ replicate 
  tests were performed with sample size $n = 100$ and $B = 1000$ bootstrap 
  samples.}
  \label{fig:hist_gamma_gev_FD}
\end{figure}

It is also possible to apply our procedure to other families of distributions. 
The data in Figure~\ref{fig:hist_gamma_gev_FD} is generated from the gamma and 
generalized extreme value distributions. The results are favorable and show that 
the procedure outlined as a remedy for fitted parameters and spatially dependent 
can be adapted for various families of distributions.


\section{Conclusion}
\label{sec:conclusion}

The KS test has base assumptions that the hypothesized distribution is 
completely specified and the data is independent. When these assumptions are 
violated, the test is no longer accurate and remedies must be performed. In the 
case of fitted parameters, parametric and non-parametric bootstrap can restore 
the size of the test. A bias correction is required if the non-parametric form 
is used \citep{Babu}. In the case of \eds{serially} dependent data, a procedure 
using bootstrap 
with copulas to model dependency shows positive results. When both assumptions 
are violated, i.e., where the data has \eds{serial} dependence and parameters 
must 
be fitted, an adjusted copula procedure also shows positive results. The tests 
appear effective for a variety of families of distributions. The copula remedy 
is not a complete solution and has limitations. Regardless of the true 
dependence, we assume an AR(1) dependence structure. Therefore, if the AR(1) 
dependence structure is a close approximation of the truth, the approach can work. 
However, in cases such as AR(2), if the approximation is too far from the true 
dependence structure the approach does not completely remedy the issue. As well, 
tests were only performed with the normal copula. It is possible that other 
copulas could provide stronger results based on the true dependence of the data.

\bibliographystyle{chicago}
\bibliography{citations.bib}


\end{document}
